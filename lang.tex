% !TeX program = xelatex

\documentclass[12pt]{article}
%\usepackage[utf8]{inputenc}
%\usepackage{tipa}
%\newcommand{\ipa}{\textipa}
\usepackage{fontspec}
\setmainfont{Linux Libertine O}
% \setlength{\parindent}{2em}
\setlength{\parskip}{1em}

\begin{document}
	\title{Adellian: Language}
	\maketitle
	
	\section{Preamble}
	
	
	
	\tableofcontents
	
	\section{General information}
	
	\section{Phonology}
	
	\subsection{Consonants}
	
	\begin{tabular}{||c | c c c c c c c||}
		Consonants & Labial & Dental & Alveolar &
		Retroflex & Palatal & Velar & Glottal \\
		\hline
		Nasal & m & \multicolumn{2}{c}{n} & & ɲ & ŋ & \\
		Stops \& Affricates & p & t & d & ʈʂ & c & k & \\
		Fricatives & f & θ & s & ʂ & ç & & h \\
		Continuants & ʋ & ð & z & ʐ & j & & ɦ \\
		Liquids & & ɫ & r & ɻ & lʲ & & \\		  
	\end{tabular}

	Adellian has a wide consonant inventory of 26 distinct sounds in total. One of its prominent feature is the way dental and alveolar sounds works there. Dental sounds tend to be more lax in Adellian, while alveolar ones are more tense and have some additional articulation feature in comparison to the dental one. \emph{n} can go either way depending on its surroundings.
	
	While having some contrast we would distinguish mainly by voicedness, in Adellian it is an additional feature rather than the main one. For example, with \emph{s/z}, it's more important that \emph{z} is a \emph{continuant} that, in certain cases, might come close to \emph{ɹ}.
	
	\emph{r} is single-vibration, and may be realized as a tap (\emph{ɾ}) for many speakers.
	
	\subsection{Vowels}
	
	\begin{tabular}{|| c | c c c c }
		Vowels & Front unrounded & Front rounded & Central & Back \\
		\hline
		High & i & y & ɪ & u \\
		Mid & e & ø & ə & o \\
		Low & ɛ & & ä & \\
		Dipthongs & \multicolumn{4}{c}{aɪ ɛɪ oɪ aʊ̯} \\
	\end{tabular}

	Vowel inventory in Adellian is higher than average, and consists of 10 distinct phonemes. However, emph{ə} (shwa) exists as an allophone that appears sometimes in so-called "weak" syllables; it is indicated in writing, but it doesn't play a phonemic difference anywhere.
\end{document}