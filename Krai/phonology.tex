% !TeX program = xelatex

\documentclass[12pt]{article}
\usepackage{fontspec}
\usepackage{xcolor}
\usepackage{verbatim}
\setmainfont{Linux Libertine O}
\setlength{\parindent}{0em}
\setlength{\parskip}{1em}

\definecolor{light-gray}{gray}{0.89}

%usage: example, gloss, translation.
\newcommand{\example}[3]{
	\colorbox{light-gray}{
		\parbox{5in}{
			\emph{Ex.: #1}\\
				  \emph{#2}\\
				  #3
		  }
	}
}

\begin{document}
	\title{Krainova: the language}
	\author{quantumFeline}
	\maketitle

	\section{Preamble}

	\tableofcontents

	\section{General information}
	
	

	\section{Phonology}

	\subsection{Consonants}

	\begin{tabular}{||c | c c c c c c c ||}
		\hline
		Consonants & Labial & Alv. & Alv.-Pal. &
		Post-alv. & Palatal & Velar & Glottal \\
		\hline
		Nasals & m & n & & & ɲ & & \\
		Affricates (v) & & dz &  dzʲ & dʒ & & & \\
		Affricates (u) & & ts & tsʲ & ʈʃ & & & \\
		Stops (v) & b & d & & & ɟ & g & \\
		Stops (u) & p & t & & & c & k & \\
		Fricatives & f & s & sʲ & ʃ & ɕ & & h \\
		Continuants & ʋ & z & zʲ & ʒ & j & & ɦ \\
		Liquids & & r & rʲ & & lʲ & ɫ & \\
		\hline
	\end{tabular}
	
	\begin{comment}
	\textasciitilde w
	\begin{tabular}{|| c | c | c | c | c | c | c | c ||}
	\hline
		& Labial & \multicolumn{2}{c|}{ Dental/ Alveolar} &
		Post-alv. & Palatal & Velar & Glottal \\
		& & Hard & Soft & & & & \\
	\hline
	
	Nasal & m & n & nʲ & & & & \\
	Affricates & & dz ts & dzʲ tsʲ & dʒ  ʈʃ & & & \\
	Stops & p b & d t & dʲ tʲ & & & g k & \\
	Fricatives & & z s & zʲ sʲ & ʒ  ʃ & ç & & h ɦ \\
	Approximants & ʋ & ɫ & lʲ & & j & & \\
	Trill & & r & rʲ & & & &\\
	\hline
	
	\end{tabular}
	\end{comment}

	\subsection{Vowels}

	\begin{tabular}{|| c | c c c || }
		\hline
		Vowels & Front unrounded & Central & Back \\
		\hline
		High & i & ɪ & u \\
		Mid & e &  & o \\
		Low & ɛ & a & \\
		Dipthongs & \multicolumn{3}{c||}{aɪ ɛɪ oɪ aʊ̯} \\
		\hline
	\end{tabular}

	\subsection{Phonotactics, allophony, and stress}

	Stress is non-phonemic and somewhat weak, mostly quantitative. Generally, it falls on the penultimate syllable, provided it's a part of the word root; otherwise, it falls on the last syllable of the root. It does not cause vowel reduction.

	TODO: syllable structure

	\section{Orthography}
	
	\subsection{Romanisation} 

	\begin{tabular}{||c | c c c c c c c ||}
		\hline
		Consonants & Labial & Alv. & Alv.-Pal. &
		Post-alv. & Palatal & Velar & Glottal \\
		\hline
		Nasals & m & n & & & ň & & \\
		Affricates (v) & & dz &  dz' & dž & & & \\
		Affricates (u) & & ts & ts' & č & & & \\
		Stops (v) & b & d & & & ď & q & \\
		Stops (u) & p & t & & & ť & k & \\
		Fricatives & f & s & s' & š & x & & h \\
		Continuants & v & z & z' & ž & j & & g \\
		Liquids & & r & ř & & ľ & l & \\
		\hline
	\end{tabular}

	\begin{tabular}{|| c | c c c || }
		\hline
		Vowels & Front unrounded & Central & Back \\
		\hline
		High & i & y & u \\
		Mid & e &  & o \\
		Low & ä & a & \\
		Dipthongs & \multicolumn{3}{c||}{ai ei oi aw} \\
		\hline
	\end{tabular}

	\section{Pronouns}

	\begin{tabular}{|| c | c | c | c | c | c ||}
		\hline
		0 & I & II & I + II & III, animate & III, inanimate\\
		\hline
		la & mi & tu & mito & ne & sy\\
		\hline
	\end{tabular}

	Basic pronouns present in Adellian are shown in the table above. They are declined by case and number, like any other noun, but typically do not require any article.
	
	Zero pronoun is used when the subject is abstract, somewhat like ``one" or ``generic you" in English. Additionally, it case serve as a placeholder pronoun when a word root is needed grammatically, but isn't present, meaning-wise.

	Demonstratives are distinguished by three degrees of proximity.

	\begin{tabular}{|| c | c | c ||}
		\hline
		Proximal & Medial & Distal \\
		\hline
		mes & täs &  nay \\
		\hline
	\end{tabular}

	\section{Nouns}
	
	Adellian nouns decline in the following way:
	
	\emph{root + article + number + case [+ locative postfix]}
	
	\example {kassaqdes}
			 {kassa + NEG + DET + ACC}
		 	 {whom? - the majority of cats}
	

\end{document}